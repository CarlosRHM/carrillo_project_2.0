\Abstract{En este proyecto hacemos una revisión miltidisciplinar, revisando desde las primeras ideas y los fundamentos teóricos que dieron origen a las redes neuronales hasta las aplicaciones que podemos encontrar en la actualidad.

Comenzamos examinando los sistemas complejos y su relación con la mecánica estadística para entender los comportamientos emergentes en estructuras interconectadas. A continuación, comparamos las redes neuronales artificiales y biológicas, destacando sus similitudes y diferencias, presentamos la Teoría de Grafos como herramienta matemática para modelar redes neuronales, seguida de un análisis de algunos de los tipos de redes más importantes. Por último, se presentan varias aplicaciones de estas redes en campos como la inteligencia artificial y la meicina, lo que demuestra su impacto en distintas disciplinas.
}