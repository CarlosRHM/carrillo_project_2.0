\section{Conclusiones}
A través de este estudio, hemos observado cómo la estructura de las redes neuronales y si dinámica pueden modelarse y analizarse utilizando herramientas de distintas disciplinas. La comparación entre redes artificiales y biológicas nos da mas claridad en los retos y oportunidades que nos da la inteligencia artificial.

La exploración de los distintos tipos de de redes demuestran la importancia de la conectividad para determinar el comportamiento de una red. Además, las diversas aplicaciones de estas redes que abarcan desde el aprendizaje automático hasta los sistemas biológicos y sociales dan indicios de su uso para nuevas y mejores tecnologías.

Las redes neuronales son una gran herramienta para comprender sistemas complejos y ofrecen nuevas técnicas en la investigación teórica y la aplicada. Los futuros avances en este campo profundizarán nuestra comprensión de sistemas más complejos y mas realistas, proporcionado el camino para soluciones innovadoras a problemas del mundo real.